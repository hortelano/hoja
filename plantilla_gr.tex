\documentclass[12pt]{article}

% aprovechamiento de la p\'agina -- fill an A4 (210mm x 297mm) page

% Note: 1 inch = 25.4 mm = 72.27 pt
% 1 pt = 3.5 mm (approx)

% vertical page layout -- one inch margin top and bottom

\topmargin     -14 mm  % top margin less 1 inch
\headheight     0 mm  % height of box containing the head
\headsep        0 mm  % space between the head and the body of the page
\textheight   276 mm
\footskip       7 mm  % distance from bottom of body to bott om of foot

% horizontal page layout -- one inch margin each side
\oddsidemargin    0 mm   % inner margin less one inch on odd pages
\evensidemargin   0 mm   % inner margin less one inch on even pages
\textwidth      159.2 mm % normal width of text on page

\usepackage{arabtex}
\usepackage{atrans}
\usepackage{nashbf}
\usepackage{twoblks}
\usepackage[utf8]{inputenc}
\usepackage[absolute]{textpos}
\usepackage{graphicx}

\usepackage{tikz,everypage}
\usetikzlibrary{backgrounds}
\newcommand*{\AbsolutePosition}[3]{%
    % #1 = x (from south west corner of page)
    % #2 = y
    % #3 = content
    \AddThispageHook{%
        \begin{tikzpicture}[remember picture,overlay]
            \draw (current page.south west) ++ (#1,#2) node[opacity=0.7] {#3};%
        \end{tikzpicture}
    }
}

%%%%%%%%%%%%%%%%%%%%%%%%%%%%%%%%%%%%%%%%%%%%%%%%
%% cómo generar el código QR
%% compilar con:  pdflatex -shell-escape <file>
%% la primera vez para generar el pdf y luego 
%% comentar.
%%%%%%%%%%%%%%%%%%%%%%%%%%%%%%%%%%%%%%%%%%%%%%%%
% \usepackage{pst-barcode}
% \usepackage{auto-pst-pdf}   
%%%%%%%%%%%%%%%%%%%%%%%%%%%%%%%%%%%%%%%%%%%%%%%%%

%% selección de la fuente global
\usepackage[
            light,
            math]
           {kurier}
\usepackage[T1]{fontenc}

\newenvironment{parrafo}{\begin{trivlist}\item []}{\end{trivlist}}
\pagestyle{empty}
\tracingarab=0
\setarab
\settrans{spanish}
\novocalize
\arabtrue

\pagestyle{empty}

\begin{document}

% %\begin{textblock*}{290mm}(185mm,258mm)
% \begin{textblock*}{290mm}(25mm,2mm) % Granada y Motril
% %\begin{textblock*}{290mm}(25mm,4mm) % Sevilla
%   \includegraphics[scale=1]{taqwim.pdf}
% \end{textblock*}

\footnotesize \twoblocks{Parte de los momentos del
  \arabfalse\transtrue \RL{.salAT} correspondiente al mes de
  \arabfalse\transtrue\settransfont{\bf\itshape} 
  mesArabTrans 
  del a\~no \textbf{agnoa h.} (agnom/agnop d.J.) para la
  ciudad de \textbf{Granada} y cercan\'{\i}as.  }{
  \begin{arabtext}
    \noindent
    .hi.s.saTu 'awqAti a.s-.salATi li^sahri \setnashbf 
    mAAG
    \setnash min `Ami \setnashbf <\bf agnoa> h- . \setnash(
    \LR{agnom}/\LR{agnop} m- .) limadInaTi \setnashbf .garnA.taTa
    \setnash wa-mA ^gAwarahA.
  \end{arabtext}}

\normalsize

\begin{parrafo}
\centerline{
\begin{tabular}{|c|c|c|c|c|c|c|c|c|c|}
\hline
\RL{al-`i^sA'}&
\RL{al-ma.grib}&
\RL{al-`a.sr}&\RL{a.z-.zuhr}&
\RL{al-^surUq}&\RL{al-fa^gr}&
\multicolumn{2}{|c|}{\RL{yawmu al-'usbU`}}&
\RL{mesSolarArab2} /\RL{mesSolarArab1}&
\RL{mesArabArab}\\
\hline
\arabfalse\transtrue \RL{`i^sA'} &
\arabfalse\transtrue \RL{ma.grib}&
\arabfalse\transtrue \RL{`a.sr}  &
\arabfalse\transtrue \RL{.zuhr}  &
\arabfalse\transtrue \RL{^surUq} &
\arabfalse\transtrue \RL{fa^gr}  &
                      \multicolumn{2}{|c|}{d\'{\i}a semana}&
                      mesSolarEsp1/mesSolarEsp2  &
\settransfont{\bf\itshape}\arabfalse\transtrue
mesArabCaja\\
\hline
horario
\end{tabular}
}

% Lugares de interés
                                         
% Güera
% s-m 1h 21m 22s
% m-a 1d 07h 57m
% Cape Town
% s-m 1h 00m 15s
% m-a 1d 05h 08m

% informe mes anterior:
                                        
\centerline{                             
\begin{tabular}{|c|c|c|c|c|c|}           
  \hline                                   
  \multicolumn{6}{|c|}{Algunos par\'ametros lunares en \textbf{Granada} para
    final de \arabfalse\transtrue\settransfont{\bf\itshape} 
  mesArabTrans}\\
  \hline
  d\'{\i}a de observar&
  puesta de Sol&
  puesta de Luna&
  dif. Luna-Sol&
  edad lunar&
  ?`avistamiento?\\
  \hline
  diadata&
  ssdata&
  msdata&
  smdata&
  madata&
  avdata\\ 
  \hline
\end{tabular}
}
 \vspace{2mm}

 \footnotesize \noindent \textbf{Granada} (37º11'N/3º36'O): longitud
 de referencia 15ºE, zona horaria +1 h, altitud sobre el nivel del mar
 828 m, \'angulo de crep\'usculo 18º, \arabfalse\transtrue \RL{qiblaT}
 100.4º, declinaci\'on dedata.

\end{parrafo}

 % \scriptsize
 % \begin{parrafo}
 % \centerline{\textsc{Carnicer\'{\i}a Albaic\'{\i}n-Halal},
 % \textit{Calderer\'{\i}a Vieja, 16 -- 18010
 % Granada -- Tel. 958 227 074}}
 % \end{parrafo}

 % \scriptsize
 % \begin{parrafo}
 % \centerline{\textsc{Carnicer\'{\i}a Tangerino
 % "Halal". Pasteler\'{\i}a \'{A}rabe},
 % \textit{Avd. Pulianas, 42 -- 18011
 % Granada -- Tel. 663 489 100}}
 % \end{parrafo}

 % \scriptsize
 % \begin{parrafo}
 % \centerline{\textsc{Carnicer\'{\i}a Elvira-Halal},
 % \textit{Elvira, 121 -- 18010
 % Granada -- Tel. 958 201 958}}
 % \end{parrafo}

 \scriptsize
 \begin{parrafo}
 \centerline{\textsc{Sabores de Al-Andalus (Natura Morisca)},
 \textit{Acera del Darro, 2 (Pasaje Regina) -- 18005
 Granada -- Tel. 627 878 242}}
 \end{parrafo}

%  \begin{parrafo}
%  \centerline{\texttt{Segunda Edición}}
%  \end{parrafo}
\end{document}
 